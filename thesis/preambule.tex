% \usepackage[latin1]{inputenc}
\usepackage[utf8]{inputenc}
\usepackage[T1]{fontenc}

%% loading this first to avoid clash with bidi/arabic
\usepackage{xunicode}
%%% For language switching -- like babel, but for xelatex
\usepackage{polyglossia}
\usepackage{csquotes}
\providecommand{\bsc}[1]{\textsc{#1}}

\setmainlanguage{english}
\setotherlanguages{arabic, french}

% define fonts for other languages
\newfontfamily\arabicfont[Script=Arabic]{Noto Naskh Arabic}

\usepackage{booktabs}
\usepackage{tabularx} 
\usepackage{multirow}
\usepackage{stix,bbding,pifont,utfsym,fontawesome}
\usepackage{subfig}


\usepackage{wrapfig}
\usepackage{graphicx}
\usepackage{here}
\graphicspath{{images/}}

\usepackage{minitoc}
\setcounter{minitocdepth}{1}
% \mtcselectlanguage{french}
%\nomtcpagenumbers  % remove page numbers from minitocs
%\nomtcrule         % removes rules = horizontal lines
%\undotted          % removes just the dots

\usepackage{fancybox}% pour fbox et shadowbox

\usepackage{setspace}%pour l'environnement onehalfspace
\onehalfspacing

\usepackage{eso-pic}
\newcommand \AlCentroPagina [1]{%
\AddToShipoutPicture *{\AtPageCenter {%
\makebox (0,0){\includegraphics %
[width=0.98\paperwidth]{#1}}}}}

\usepackage[top=3cm, bottom=3cm, left=3cm, right=3cm]{geometry} %pour les marges
%\usepackage[top=3.5cm, bottom=3.5cm, left=3.5cm, right=3.5cm]{geometry}

\usepackage{epigraph} %pour les citation dans les débuts des chapitres
\setlength{\epigraphwidth}{8cm}
\renewcommand{\epigraphsize}{\footnotesize}

\usepackage [Lenny]{fncychap}%pour modifier l'allure des chapitres

\usepackage{fancyhdr}
\pagestyle{fancy}
\lhead{\leftmark}% gère le coin gauche de l’en-tête.
\chead{}% gère le centre de l’en-tête.
\rhead{}% gère le coin droit de l’en-tête.
\lfoot{}% gère le coin gauche du pieds de page.
\cfoot{}% gère le centre du pieds de page.
\rfoot{\thepage}% gère le coin droit du pieds de page.
\renewcommand{\chaptermark}[1]{\markboth{\bsc{\chaptername~\thechapter{} :} #1}{}}

\usepackage{url} %pour écrire des adresses cliquables

\usepackage{amsthm}
\usepackage{amssymb}
\usepackage{mathrsfs}
\usepackage{amsmath}%le package pour faire des formules et équations perfectionnées
\newtheorem{derty}{Définition}%[section]
\newtheorem{lem}{Lemme}%[section]
\newtheorem{thm}{Théorème}%[section]
\newtheorem{cor}{Corollaire}%[section]
\newtheorem{pro}{Proposition}%[section]
%definition
\theoremstyle{definition}
\newtheorem{definition}{Definition}[section]

\usepackage{color}
\usepackage{colortbl}
\definecolor{marron}{rgb}{0.65,0.16,0.16}
\definecolor{grisb}{rgb}{0.86,0.86,0.86}
\newcolumntype{M}[1]{>{\centering \scriptsize}p{#1}}
\newcolumntype{K}[1]{>{\centering \footnotesize}p{#1}}
%\usepackage{array}



\usepackage{hyperref} 
\hypersetup{
backref=true,       
pagebackref=true,   
colorlinks=true, 
breaklinks=true,
urlcolor= blue, 
linkcolor= black, 
citecolor=blue, 
pdftitle={Engineering Thesis}, 
pdfauthor={Moustafa Ahmed}, 
pdfsubject={Design and Implementation of an Educational Cryptocurrency Blockchain System and Consensus Algorithm} 
}

\title{\huge \textsc{\textbf Engineering Thesis}}
\author{\large Moustafa Ahmed} 
\date{2021}



\usepackage{listings}
\usepackage{multirow}

\usepackage{slashbox}

\usepackage{adjustbox}
\usepackage{blindtext}

\usepackage{tikz}
\usetikzlibrary{calc}

\usepackage[french]{algorithm2e}

\lstnewenvironment{pascal}
{
\lstset
			{
			language=Pascal,
			keywordstyle=\bfseries\color{marron},
			stringstyle=\it\color{blue},
			basicstyle=\rm\footnotesize,
			showstringspaces=false,
			%numbers=left,
			numberstyle=\tiny\bfseries,
			stepnumber=1,
			numberfirstline=false,
			%linewidth=9cm,
			aboveskip=3mm,%marge avant le listing
			belowskip=3mm,%marge après le listing
			xrightmargin=5mm,%marge à droite du listing
			xleftmargin=5mm,%marge à gauche du listing
			%float=!h,
			frame=tblr,%tBlR
			rulesep=0.5mm,%la distance entre les traits des bordures doubles
			framesep=2mm,%distance entre le code et la bordure
			frameround=ftft,%framerule=2pt,%l'épaisseure de la bordure
			rulecolor={\color{marron}},
			rulesepcolor={\color{white}},
			captionpos=b,
			backgroundcolor=\color{grisb}
			}
}
{}
